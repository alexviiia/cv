%%%%%%%%%%%%%%%%%%%%%%%%%%%%%%%%%%%%%%%%%
% Compact Academic CV
% LaTeX Template
% Version 1.0 (10/6/2012)
%
% This template has been downloaded from:
% http://www.LaTeXTemplates.com
%
% Original author:
% Dario Taraborelli (http://nitens.org/taraborelli/home)
%
% License:
% CC BY-NC-SA 3.0 (http://creativecommons.org/licenses/by-nc-sa/3.0/)
%
% Important:
% This template needs to be compiled using XeLaTeX
%
% Note: this template has the option to use the Hoefler Text font, see the
% font configurations section below for instructions on using this font
%
%%%%%%%%%%%%%%%%%%%%%%%%%%%%%%%%%%%%%%%%%

%----------------------------------------------------------------------------------------
%	PACKAGES AND OTHER DOCUMENT CONFIGURATIONS
%----------------------------------------------------------------------------------------

\documentclass[12pt]{article} % Document font size and paper size

\usepackage{fontspec} % Allows the use of OpenType fonts

%\usepackage{geometry} % Allows the configuration of document margins
%\geometry{textwidth=5.5in, textheight=8.5in, marginparsep=7pt, marginparwidth=.6in} % Document margin settings
\usepackage[margin=1in,top=1in, bottom=1in]{geometry}
\setlength\parindent{0in} % Remove paragraph indentation

\usepackage[usenames,dvipsnames]{xcolor} % Custom colors

\usepackage{sectsty} % Allows changing the font options for sections in a document
\usepackage[normalem]{ulem} % Custom underlining
\usepackage{xunicode} % Allows generation of unicode characters from accented glyphs
\defaultfontfeatures{Mapping=tex-text} % Converts LaTeX specials (``quotes'' --- dashes etc.) to unicode

\usepackage{marginnote} % For margin years
\newcommand{\years}[1]{\marginnote{\scriptsize #1}} % New command for including margin years
\renewcommand*{\raggedleftmarginnote}{}
\setlength{\marginparsep}{7pt} % Slightly increase the distance of the margin years from the contant
\reversemarginpar

\usepackage[xetex, bookmarks, colorlinks, breaklinks, pdftitle={Alejandro Ochoa CV},pdfauthor={Alejandro Ochoa}]{hyperref} % PDF setup - set your name and the title of the document to be incorporated into the final PDF file meta-information
\hypersetup{linkcolor=blue,citecolor=blue,filecolor=black,urlcolor=MidnightBlue} % Link colors

% MATH stuff (limited)
\usepackage{xspace}
\usepackage{amsmath,amsfonts,amssymb}
\newcommand{\Fst}{\ensuremath{F_{\text{ST}}}\xspace}

%----------------------------------------------------------------------------------------
%	FONT CONFIGURATIONS
%----------------------------------------------------------------------------------------

% Linux Libertine Font (default)
\setromanfont [Ligatures={Common}, Numbers={OldStyle}, Variant=01]{Linux Libertine O} % Main text font
%\setmonofont[Scale=0.8]{Monaco} % Set mono font (e.g. phone numbers)
\sectionfont{\mdseries\upshape\Large} % Set font options for sections
\subsectionfont{\mdseries\scshape\normalsize} % Set font options for subsections
\subsubsectionfont{\mdseries\upshape\large} % Set font options for subsubsections
\chardef\&="E050 % Custom ampersand character

%----------------------------------------------------------------------------------------

\begin{document}

\begin{center}
  {\LARGE Alejandro Ochoa}\\[0.5cm] % Your name
  Duke Center for Statistical Genetics and Genomics\\
  Department of Biostatistics and Bioinformatics\\
  Fitzpatrick Center for Interdisciplinary Engineering, Medicine and Applied Sciences (CIEMAS)\\
  101 Science Dr, Room 2177C\\
  Durham, NC 27705\\
  \href{mailto:alejandro.ochoa@duke.edu}{alejandro.ochoa@duke.edu}\\ % Your email address
  \url{https://ochoalab.github.io/} \\ % http://viiia.org/research % Your academic/personal website
\end{center}

%\vfill % Whitespace between contact information and specific CV information

%\section*{Research Interests}
%Statistics, Bioinformatics, Computational Biology\\ % Your primary areas of research interest
%Population Genetics\\
%Protein Sequence Evolution and Function\\
%Human, Malaria Parasites

\section*{Education}

\years{2013}
PhD in Molecular Biology, Princeton University
\\
Dissertation:
\emph{Protein domain prediction using context statistics, the false discovery rate, and comparative genomics, with application to Plasmodium falciparum}
\\
Advisers:
Mona Singh (Computer Science) and Manuel Llinás (Molecular Biology)
\\
\years{2006}
\textsc{BS} in Biology and Mathematics, Massachusetts Institute of Technology

\section*{Awards \& Honors}

\years{2022}
Lathisms: showcase of contributions of Latinx and Hispanic mathematicians during Hispanic Heritage Month.
\url{https://www.lathisms.org/calendars/calendar-2022}
\\
\years{2020}
Whitehead Scholar, Whitehead Charitable Foundation.
For Duke junior faculty with exceptional potential for research and research training in the biomedical sciences.
\\
\years{2008}
NSF Graduate Research Fellowship.
Computational Biology.
\\
\years{2008}
Ford Foundation Diversity Fellowship Predoc. Comp.
(Declined, listed as Honorable Mention.)
\\
\years{2006}
MIT Department of Biology Merck Prize.
Awarded to a graduating senior for outstanding research and academic performance in biophysics or bioinformatics.
\\
\years{2001}
Mexican Math Olympiad Gold Medal.
Awarded to top 15 competitors nationally.

\section*{Employment}

\years{2018-now}\emph{Assistant Professor},
Duke Center for Statistical Genetics and Genomics, 
Department of Biostatistics and Bioinformatics,
Duke University\\
\years{2013-2018}\emph{Postdoctoral Research Associate} at John D Storey's group,
Lewis-Sigler Institute for Integrative Genomics, and
Center for Statistics and Machine Learning,
Princeton University

\section*{Publications}

% AUTOMATICALLY GENERATED by pubsMkTex.pl!  DO NOT EDIT!

\subsection*{Journal articles}
\years{2021}\textbf{Alejandro Ochoa}, John D Storey. Estimating \Fst and kinship for arbitrary population structures. \textit{PLoS Genet}. 17(1) e1009241. PMID \href{https://www.ncbi.nlm.nih.gov/pubmed/33465078}{33465078}. \\
\years{2017}\textbf{Alejandro Ochoa}, Mona Singh. Domain prediction with probabilistic directional context. \textit{Bioinf}. 33(16) 2471-8. PMID \href{https://www.ncbi.nlm.nih.gov/pubmed/28407137}{28407137}. \\
\years{2016}Simon A Cobbold, Joana M Santos, \textbf{Alejandro Ochoa}, David H Perlman, Manuel Llinás. Proteome-wide analysis reveals widespread lysine acetylation of major protein complexes in the malaria parasite. \textit{Sci Rep}. 2016;6:19722. PMID \href{https://www.ncbi.nlm.nih.gov/pubmed/26813983}{26813983}. \\
\years{2015}\textbf{Alejandro Ochoa}, John D Storey, Manuel Llinás, Mona Singh. Beyond the \emph{E}-value: stratified statistics for protein domain prediction. \textit{PLoS Comput Biol}. 11 e1004509. PMID \href{https://www.ncbi.nlm.nih.gov/pubmed/26575353}{26575353}. \\
\years{2013}Moriah L Szpara, Derek Gatherer, \textbf{Alejandro Ochoa}, Benjamin Greenbaum, Aidan Dolan, Rory J Bowden, Lynn W Enquist, Matthieu Legendre, Andrew J Davison. Evolution and diversity in human herpes simplex virus genomes. \textit{J Virol}. 88:1209-27. PMID \href{https://www.ncbi.nlm.nih.gov/pubmed/24227835}{24227835}. \\
\years{2011}\textbf{Alejandro Ochoa}, Manuel Llinás, Mona Singh. Using context to improve protein domain identification. \textit{BMC Bioinformatics}. 12:90. PMID \href{https://www.ncbi.nlm.nih.gov/pubmed/21453511}{21453511}. \\
\years{2007}Gevorg Grigoryan, \textbf{Alejandro Ochoa}, Amy E Keating. Computing van der Waals energies in the context of the rotamer approximation. \textit{Proteins}. 68(4) 863-78. PMID \href{https://www.ncbi.nlm.nih.gov/pubmed/17554777}{17554777}. 

\subsection*{Manuscripts in Submission}
\years{2021}Brian I Shaw, \textbf{Alejandro Ochoa}, Cliburn Chan, Chloe Nobuhara, Rasheed Gbadegesin, Annette M Jackson, Eileen T Chambers. HLA Loci and Recurrence of Focal Segmental Glomerulosclerosis In Pediatric Kidney Transplantation. Submitted. \\
\years{2020}Young-Sook Kim, Graham D Johnson, Jungkyun Seo, Alejandro Barrera, William H Majoros, \textbf{Alejandro Ochoa}, Andrew S Allen, Timothy E Reddy. Detecting regulatory elements in high-throughput reporter assays. Preprint: {\small \url{https://doi.org/10.1101/2020.08.07.241901}} \\
\years{2019}Yiqi Yao, \textbf{Alejandro Ochoa}. Testing the effectiveness of principal components in adjusting for relatedness in genetic association studies. Preprint: {\small \url{https://doi.org/10.1101/858399}} \\
\years{2019}\textbf{Alejandro Ochoa}, John D Storey. New kinship and \Fst estimates reveal higher levels of differentiation in the global human population. Preprint: {\small \url{https://doi.org/10.1101/653279}} \\
\years{2016}\textbf{Alejandro Ochoa}, John D Storey. \Fst and kinship for arbitrary population structures I: Generalized definitions. Preprint: {\small \url{https://doi.org/10.1101/083915}} 

\subsection*{Acknowledgments}
\years{2019}Irineo Cabreros, John D Storey. A Likelihood-Free Estimator of Population Structure Bridging Admixture Models and Principal Components Analysis. \textit{Genetics}. 212(4) 1009-29. PMID \href{https://www.ncbi.nlm.nih.gov/pubmed/31028112}{31028112}. \\
\years{2016}Prem Gopalan, Wei Hao, David M Blei, John D Storey. Scaling probabilistic models of genetic variation to millions of humans. \textit{Nat Genet}. 48(12) 1587-90. PMID \href{https://www.ncbi.nlm.nih.gov/pubmed/27819665}{27819665}. \\
\years{2014}Anton V Persikov, Mona Singh. De Novo Prediction of DNA-Binding Specificities for Cys2His2 Zinc Finger Proteins. \textit{Nucleic Acids Res}. 42(1) 97-108. PMID \href{https://www.ncbi.nlm.nih.gov/pubmed/24097433}{24097433}. 
 % generated automatically!

\section*{Distributed software}

% AUTOMATICALLY GENERATED by softwareMkTex.pl!  DO NOT EDIT!

\subsection*{Major packages}
\years{2021-2023}simfam: Simulate and Model Family Pedigrees With Structured Founders. R, C++. \\ Available on \href{https://cran.r-project.org/package=simfam}{CRAN} and \url{https://github.com/OchoaLab/simfam}.\\
\years{2019-2023}genio: Genetics Input/Output Functions. R, C++. \\ Available on \href{https://cran.r-project.org/package=genio}{CRAN} and \url{https://github.com/OchoaLab/genio}.\\
\years{2019-2023}simtrait: Simulate Complex Traits from Genotypes. R. \\ Available on \href{https://cran.r-project.org/package=simtrait}{CRAN} and \url{https://github.com/OchoaLab/simtrait}.\\
\years{2017-2023}popkin: Estimate Kinship and FST under Arbitrary Population Structure. R, C++. \\ Available on \href{https://cran.r-project.org/package=popkin}{CRAN} and \url{https://github.com/StoreyLab/popkin}.\\
\years{2017-2023}bnpsd: Model and Simulate Admixed Populations. R. \\ Available on \href{https://cran.r-project.org/package=bnpsd}{CRAN} and \url{https://github.com/StoreyLab/bnpsd}.\\
\years{2014-2020}dPUC2: Domain Prediction Using Context, Version 2. Perl, C. \\ \url{https://github.com/alexviiia/dpuc2}.\\
\years{2014-2020}DomStratStats: Domain Stratified Statistics (q-values and local FDRs). Perl. \\ \url{https://github.com/alexviiia/DomStratStats}.\\
\years{2014-2019}RandProt: High-order Markov random models for protein sequences. Perl. \\ \url{https://github.com/alexviiia/RandProt}.

\subsection*{Minor packages}
\years{2021-2023}simgenphen: Simulate Genotypes and Phenotypes. R. \\ \url{https://github.com/OchoaLab/simgenphen}.\\
\years{2021-2023}genbin: R wrappers for binaries in genetics. R. \\ \url{https://github.com/OchoaLab/genbin}.\\
\years{2020-2023}ligera: LIght GEnetic Robust Association. R, C++. \\ \url{https://github.com/OchoaLab/ligera}.\\
\years{2019-2022}popkinsuppl: Supplement to "popkin" package. R. \\ \url{https://github.com/OchoaLab/popkinsuppl}.

\subsection*{Paper repositories}
\years{2021-2023}bias-assoc-paper: Kinship bias association project. R, bash, LaTeX, markdown. \\ \url{https://github.com/OchoaLab/bias-assoc-paper}.\\
\years{2019-2023}pca-assoc-paper: PCA association project. R, bash, LaTeX, markdown. \\ \url{https://github.com/OchoaLab/pca-assoc-paper}.\\
\years{2019-2023}data: Instructions for real data processing shared across projects. R, bash, Perl, markdown. \\ \url{https://github.com/OchoaLab/data}.\\
\years{2019}human-differentiation-manuscript: Human differentiation analysis. R, bash, markdown. \\ \url{https://github.com/StoreyLab/human-differentiation-manuscript}.
 % generated automatically!

\section*{Invited Talks}

% AUTOMATICALLY GENERATED by talksMkTex.pl!  DO NOT EDIT!

\years{2018}Population Biology seminar, Department of Biology, Duke University, Durham, NC. 2018-09-27.\\
\years{2018}Department of Biostatistics and Bioinformatics, Duke University, Durham, NC. 2018-02-21.\\
\years{2018}Department of Biostatistics, Johns Hopkins University, Baltimore, MD. 2018-01-22.\\
\years{2017}Department of Biology, University of Richmond, Richmond, VA. 2017-11-10.\\
\years{2017}\emph{Princeton Research Day}. Princeton University, Princeton, NJ. 2017-05-11.\\
\years{2017}Department of Genetics, University of North Carolina, Chapel Hill, NC. 2017-02-13.\\
\years{2016}\emph{New York Area Population Genomics Workshop 2016}. Princeton University, Princeton, NJ. 2016-01-21.\\
\years{2015}\emph{Probabilistic Modeling in Genomics conference}. Cold Spring Harbor Laboratory, Cold Spring Harbor, NY. 2015-10-15.\\
\years{2013}\emph{Telepresentation for Yun Song's group}. UC Berkeley, Berkeley, CA. 2013-04-17.\\
\years{2013}\emph{Biological sequence analysis and probabilistic models conference}. HHMI Janelia Farm, Ashburn, VA. 2013-03-25.\\
\years{2013}NCBI, NIH, Bethesda, MD. 2013-02-25.\\
\years{2012}\emph{Recruiting conference}. Department of Computer Science, Princeton University, Princeton, NJ. 2012-03-01.\\
\years{2004}Rotary Club Paso del Norte, Ciudad Juarez, CHIH, Mexico. 2004-08-26.
 % generated automatically!

\section*{Conference posters}

% AUTOMATICALLY GENERATED by postersMkTex.pl!  DO NOT EDIT!

\years{2019}Alejandro Ochoa, John D Storey. Relatedness and Differentiation in Arbitrary Population Structures. \emph{4th Mexico Population Genomics Meeting}. Amoxcalli Complex, Faculty of Sciences, University City, National Autonomous University of Mexico, Mexico City, DF, Mexico.\\
\years{2018}Alejandro Ochoa, John D Storey. Relatedness and Differentiation in Arbitrary Population Structures. \emph{Probabilistic Modeling in Genomics}. Cold Spring Harbor Laboratory, Cold Spring Harbor, NY.\\
\years{2018}Alejandro Ochoa, John D Storey. Relatedness and Differentiation in Arbitrary Population Structures. \emph{Population, Evolutionary and Quantitative Genetics Conference}. Madison Concourse Hotel, Madison, WI.\\
\years{2016}Alejandro Ochoa, John D Storey, Srikanth Gottipati, Shashank Rohatagi, Andrew Forbes, Deborah Profit, Raymond Sanchez, Margaretta Nyilas, William Carson. Mixed Effects Modeling of Placebo Response Across 8 Placebo-controlled Aripiprazole Trials Spanning 20 Years in Acutely Relapsed Schizophrenia Patients. \emph{Neuroscience Education Institute Psychopharmacology Congress}. Broadmoor Convention Center, Colorado Springs, CO.\\
\years{2016}Srikanth Gottipati, Alejandro Ochoa, Shashank Rohatagi, Andrew Forbes, Deborah Profit, Raymond Sanchez, Margaretta Nyilas, William Carson. Canonical Loadings of PANSS Subscales Show Differential Placebo and Aripiprazole Drug Responses in Schizophrenia Patients. \emph{Neuroscience Education Institute Psychopharmacology Congress}. Broadmoor Convention Center, Colorado Springs, CO.\\
\years{2015}Alejandro Ochoa, John D Storey. \Fst generalized for arbitrary population structures. \emph{John W. Tukey 100th Birthday Celebration conference}. Center for Statistics and Machine Learning, Princeton University, Princeton, NJ.\\
\years{2013}Alejandro Ochoa, John D Storey, Manuel Llinás, Mona Singh. Forget the \emph{E}-value: family-based \emph{q}-values for protein domain prediction, and empirical error detection. \emph{Biological sequence analysis and probabilistic models conference}. HHMI Janelia Farm, Ashburn, VA.\\
\years{2010}Alejandro Ochoa, Manuel Llinás, Mona Singh. Using context to predict protein domains across diverse organisms. \emph{Recomb Systems Biology conference}. Columbia University, New York, NY.
 % generated automatically!

\section*{Teaching}

\subsection*{Duke University}
\years{2022-2024}Human Genetics. (Co-Lecturer Spring 2022-2024)\\
\years{2021-2024}UPGEN 778A-F Mini course: Genetic population structure and relatedness (Fall 2021-2024)\\
\years{2019-2024}SIBS lecture on genetic association studies (Summer 2019, 2021-2024)\\
\years{2019}BIOS 900 Current Problems in Biostatistics. Special lecture on kinship and \Fst\\
\years{2019}BIOS 710 Statistical Genetics and Genetic Epidemiology. Special lecture on kinship and \Fst

\subsection*{Princeton University}
\years{2017}Intro to Genomics and Comp Bio.  Quant Comp Bio, Comp Sci. (Co-Lecturer Fall 2017)\\
\years{2016-2017}Discussion Leader.  Summer Undergraduate Research Program (Summer 2016, 2017)\\
\years{2014-2017}Statistical Treatment of Data. Mol Bio, Quant Comp Bio. (Workshop Fall 2014-2017)\\
\years{2011}Intro to Genomics and Comp Mol Bio.  Mol Bio, Comp Sci. (TA Fall 2011)\\
\years{2008}Core Laboratory. Mol Bio. (TA Spring 2008)

\subsection*{Massachusetts Institute of Technology}
\years{2005}Calculus. OME Project Interphase (TA Summer 2005)\\
\years{2003-2004}Multivar Calculus and Diff Eqs. OME Seminar XL (Tutor Fall 2003, Spring 2004)\\
\years{2003-2004}Calculus. MITE2S Program (TA Summer 2003, 2004)

\section*{Mentoring}

% AUTOMATICALLY GENERATED by studentsMkTex.pl!  DO NOT EDIT!

\subsection*{Current Trainees - Primary Adviser}
\years{2022-now}Ratchanon "RP" Pornmongkolsuk. PhD. Graduate rotation, Dissertation adviser. Ochoa Laboratory. University Program in Genetics and Genomics, Duke University.\\
\years{2022-now}Zhuoran Hou. PhD. Graduate rotation, Dissertation adviser. Ochoa Laboratory. Department of Biostatistics and Bioinformatics, Duke University.

\subsection*{Current Trainees - Secondary Adviser}
\years{2021-now}Cymfenee Dean-Phifer. PhD. Graduate rotation, Dissertation committee. Goldberg Laboratory. Computational Biology and Bioinformatics Program, Duke University.\\
\years{2023-now}Bide "Peter" Chen. PhD. Dissertation committee. Goldberg Laboratory. University Program in Genetics and Genomics, Duke University.\\
\years{2023-now}Grace E. Rhodes. PhD. Graduate rotation, Dissertation committee. Allen Laboratory. Department of Biostatistics and Bioinformatics, Duke University.\\
\years{2023-now}Gabriel Kennedy. PhD. Dissertation committee. Goldberg Laboratory. University Program in Genetics and Genomics, Duke University.\\
\years{2023-now}Anvita Kulshrestha. PhD. Dissertation committee. Ashley-Koch Laboratory. University Program in Genetics and Genomics, Duke University.\\
\years{2024-now}Jennifer Drucker Varner. Fellow. Scholarship Oversight Committee. Gbadegesin Laboratory. Department of Pediatrics, Duke University.\\
\years{2024-now}Constantine Stavrianidis. PhD. Dissertation committee. Allen Laboratory. Computational Biology and Bioinformatics Program, Duke University.

\subsection*{Past Trainees - Primary Adviser}
\years{2019-2020}Yiqi Yao. Master's. BCTIP internship, Master's project adviser. Ochoa Laboratory. Department of Biostatistics and Bioinformatics, Duke University. Now Senior Business Analyst at BenHealth.\\
\years{2019-2022}Amika Sood. Postdoctoral. Postdoctoral adviser. Ochoa Laboratory. Department of Biostatistics and Bioinformatics, Duke University. Now Staff Scientist at Complex Carbohydrate Research Center (CCRC), The University of Georgia.\\
\years{2020-2021}Zhuoran Hou. Master's. BCTIP internship. Ochoa Laboratory. Department of Biostatistics and Bioinformatics, Duke University. Now PhD Student at Duke University, BB.\\
\years{2020-2025}Tiffany Tu. PhD. Graduate rotation, Dissertation adviser. Ochoa Laboratory. Computational Biology and Bioinformatics Program, Duke University.\\
\years{2020-2022}Jiajie Shen. Master's. Research, Master's project adviser. Ochoa Laboratory. Department of Biostatistics and Bioinformatics, Duke University.\\
\years{2021-2022}Emmanuel Mokel. Undergraduate. Research Independent Study. Ochoa Laboratory. Department of Statistical Science, Duke University.\\
\years{2023-2023}Danielle Mensah. Undergraduate. Research Independent Study. Ochoa Laboratory. Department of Computer Science, Duke University.

\subsection*{Past Trainees - Secondary Adviser}
\years{2009}Neo Christopher Chung. PhD. Graduate rotation. Llinás Laboratory. Quantitative Computational Biology Program, Princeton University. Now Adjunct Faculty at University of Warsaw.\\
\years{2010}Jeremy Bigness. PhD. Graduate rotation. Singh Laboratory. Quantitative Computational Biology Program, Princeton University. Now PhD Student at Brown University.\\
\years{2011}Sebastian Nasamu. Undergraduate. Summer project. Llinás Laboratory. Department of Molecular Biology, Princeton University. Now Post-doctoral Research Fellow at Johns Hopkins Bloomberg School of Public Health.\\
\years{2019}Yuncheng Duan. PhD. Graduate rotation. Ochoa Laboratory. Department of Biology, Duke University. Now PhD Student at Duke University, Allen Lab.\\
\years{2019-2020}Shengyu Li. Master's. Master's project committee. Allen Laboratory. Department of Biostatistics and Bioinformatics, Duke University. Now PhD Student at Duke University, CBB.\\
\years{2019-2024}Xue "Scarlett" Zou. PhD. Dissertation committee. Allen Laboratory. Computational Biology and Bioinformatics Program, Duke University.\\
\years{2020-2022}Iman Hamid. PhD. Dissertation committee. Goldberg Laboratory. University Program in Genetics and Genomics, Duke University. Now Scientist at Variant Bio.\\
\years{2020-2021}Bobby Boone IV. Master's. Master's project committee. Landstrom Laboratory. Department of Biostatistics and Bioinformatics, Duke University. Now Biostatistician II at University of Utah Health.\\
\years{2020-2023}Brandon M. Lê. PhD. Dissertation committee. Ashley-Koch Laboratory. University Program in Genetics and Genomics, Duke University.\\
\years{2020-2023}Rachel Cason. Resident physician. Scholarship Oversight Committee, Master's project committee. Gbadegesin Laboratory. Department of Pediatrics, Duke University.\\
\years{2021-2022}Valerie Gartner. PhD. Dissertation committee. Wray Laboratory. University Program in Genetics and Genomics, Duke University.\\
\years{2021-2022}Hongyu Du. Master's. Master's project committee. Allen Laboratory. Department of Biostatistics and Bioinformatics, Duke University.\\
\years{2021-2022}Krista Pipho. PhD. Dissertation committee. Goldberg Laboratory. University Program in Genetics and Genomics, Duke University.\\
\years{2021-2022}Weiliang "Frank" Tian. Master's. Master's project committee. Soderling Laboratory. Department of Biostatistics and Bioinformatics, Duke University.\\
\years{2022-2025}Changxin Wan. PhD. Dissertation committee. Ji Laboratory. Computational Biology and Bioinformatics Program, Duke University.\\
\years{2023-2024}Jinting Justin Liu. Master's. Master's project committee. Majoros Laboratory. Department of Biostatistics and Bioinformatics, Duke University.\\
\years{2024-2024}Katelyn Jaggi. PhD. Graduate rotation. Ochoa Laboratory. University Program in Genetics and Genomics, Duke University.\\
\years{2024-2024}Erick Figueroa Ildefonso. PhD. Graduate rotation. Ochoa Laboratory. University Program in Genetics and Genomics, Duke University.\\
\years{2024-2025}Elisa Ma. Master's. Master's project committee. Landstrom and Allen Laboratory. Department of Biostatistics and Bioinformatics, Duke University.\\
\years{2025-2025}Madison Strain. PhD. Dissertation committee. Ashley-Koch Laboratory. University Program in Genetics and Genomics, Duke University.
 % generated automatically!

\section*{Research experience}

\years{2018-now} Principal investigator, Duke.  I lead a team of trainees (graduate students, postdocs, master students and undergraduates) pursuing original research in statistical genetics and computational biology.  Our lab focuses on models and applications for structured populations, including multiethnic and admixed cohorts and modeling cryptic relatedness.\\
\years{2013-2018} Postdoctoral Research, Storey Lab, Princeton. Developed new definitions and tools to study arbitrary population structures. Also applied GWAS and mixed effects modeling to longitudinal psychiatric drug clinical trials.\\
\years{2007-2013} Graduate Research Thesis, Singh Lab and Llinás Lab, Princeton.  Developed probabilistic models and statistical methods that improve protein domain prediction.  Studied the \emph{Plasmodium falciparum} proteome, focusing on the AP2 transcription factors.  Used experimental techniques (cloning, protein purification and protein-binding microarrays) to test predicted AP2 domains for DNA binding.\\
\years{2007} Graduate Rotation, Tavazoie Lab, Princeton. Cloned an RNA aptamer for a phage display assay.\\
\years{2006-2007} Graduate Rotation, Singh Lab, Princeton. Analyzed protein interactions predicted from the presence of interacting domain pairs.\\
\years{2006} Graduate Rotation, Troyanskaya Lab, Princeton.  Built and analyzed a potential "gold standard" of Gene Ontology terms integrated from predictions of multiple sources, in \emph{Saccharomyces cerevisiae} and \emph{Homo sapiens}.\\
\years{2005-2006} Undergraduate Research Assistant, Keating Lab, MIT.  Analyzed full-atom computational protein designs using modified van der Waals potentials.

%\section*{Service to the profession}
%\years{2015} Reviewed a manuscript for the RECOMB 2016 conference.\\
%\years{2014} Reviewed a manuscript for PLoS Computational Biology.\\
%\years{2011} Reviewed a manuscript for BMC Bioinformatics.

\section*{Organizations}

\years{2023-now} American Mathematical Society (AMS).  Member.\\
\years{2020-now} American Society of Human Genetics (ASHG). Member.\\
\years{2006-2010} Latino Graduate Student Association, Princeton University. Board member: Technology Specialist 2007-2010 (handled website, email list, photos, calendar).\\
\years{2002-2006} Association of Puerto Rican students, MIT. Member.

\section*{Outreach, admissions, committees, and other service}

\years{2020-now}Duke Center for Combinatorial Gene Regulation: Outreach and Engagement.
Funded by a NIH Centers for Excellence in Genomics Science (CEGS) grant.
Includes the Duke Genomic Scholars Program: providing accessible genomic training for a diverse workforce.  Duke University.\\
\years{2019-now}CBB PhD Admissions Committee.  Duke University.\\
\years{2019-now}B\&B Diversity and Inclusion Committee.  Duke University.\\
\years{2024}B\&B retreat, moderated session on Research Working Group.  Duke University.\\
\years{2024}Using Race, Ethnicity \& Ancestry as Population Descriptors in Genetics and Genomics Research (NASEM report symposium), faculty panel.  Duke University.\\
\years{2023}CBB retreat, faculty panel.  Duke University.\\
\years{2023}Genomic Scholars Program, faculty panel.  Duke University.\\
\years{2023}Session moderator, 021: Human genome evolving I. ASHG annual meeting.\\
\years{2023}Abstract reviewer, topic Evolutionary and Population Genetics. ASHG annual meeting.\\
\years{2022}Session moderator, S41: Populations evolving: Modeling genetic variation to understand evolutionary processes.  ASHG annual meeting.\\
\years{2021-2022}Faculty Committee Member, Duke Next Generation Leaders.  Duke University.\\
\years{2021}B\&B Faculty Search Committee.  Duke University.\\
\years{2021}B\&B Master of Biostatistics Virtual Visit Day.  Duke University.\\
\years{2019}B\&B informational interviewing practicum.  Duke University.\\
\years{2017}Panelist at HISPA Latinos in College Conference. Princeton University.\\
\years{2010}Ivy-plus recruiting fair at University of Puerto Rico (4 campuses) for Princeton.\\
\years{2008-2011}Science and Engineering Expo. HHMI and Princeton University.\\
\years{2008}Graduate school recruiting, student group panel, I represented the Latino Graduate Student Association (LGSA).  Princeton University.\\
\years{2007}Helped local high school students with college personal statements. PUPP, Princeton University.

\section*{Languages}

English. Native proficiency.\\
Spanish. Native proficiency.\\
French. Fluent reading and writing, conversational speaking.

\vfill{} % Whitespace before final footer

%----------------------------------------------------------------------------------------
%	FINAL FOOTER
%----------------------------------------------------------------------------------------

\begin{center}
  {\scriptsize Last updated: \today} % Any final footer text such as a URL to the latest version of your CV, last updated date, compiled in XeTeX, etc
  % \- •\- \href{http://www.LaTeXTemplates.com}{http://www.LaTeXTemplates.com}
\end{center}

%----------------------------------------------------------------------------------------

\end{document}
